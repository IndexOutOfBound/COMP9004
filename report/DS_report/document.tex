             
%%%%%%%%%%%%%%%%%%%%%%%%%%%%%%%%%%%%%%%%%
% University Assignment Title Page 
% LaTeX Template
% Version 1.0 (27/12/12)
%
% This template has been downloaded from:
% http://www.LaTeXTemplates.com
%
% Original author:
% WikiBooks (http://en.wikibooks.org/wiki/LaTeX/Title_Creation)
%
% License:
% CC BY-NC-SA 3.0 (http://creativecommons.org/licenses/by-nc-sa/3.0/)
% 
% Instructions for using this template:
% This title page is capable of being compiled as is. This is not useful for 
% including it in another document. To do this, you have two options: 
%
% 1) Copy/paste everything between \begin{document} and \end{document} 
% starting at \begin{titlepage} and paste this into another LaTeX file where you 
% want your title page.
% OR
% 2) Remove everything outside the \begin{titlepage} and \end{titlepage} and 
% move this file to the same directory as the LaTeX file you wish to add it to. 
% Then add \input{./title_page_1.tex} to your LaTeX file where you want your
% title page.
%
%%%%%%%%%%%%%%%%%%%%%%%%%%%%%%%%%%%%%%%%%
%\title{Title page with logo}
%----------------------------------------------------------------------------------------
%   PACKAGES AND OTHER DOCUMENT CONFIGURATIONS
%----------------------------------------------------------------------------------------

\documentclass[12pt]{article}
\usepackage[english]{babel}
\usepackage[utf8x]{inputenc}
\usepackage{amsmath}
\usepackage{graphicx}
\usepackage[colorinlistoftodos]{todonotes}
\usepackage{setspace}

\usepackage{amssymb}
\usepackage{enumerate}
\usepackage[english]{babel}
\usepackage{caption}

\usepackage{listings}
\usepackage{color}
\usepackage[left=1.5cm, right=1.5cm, top=1.5cm, bottom=2cm]{geometry}
\usepackage{float}
\usepackage{subcaption}
\usepackage[toc,page]{appendix}
\setlength\parindent{0pt}
\usepackage{subcaption}

\usepackage{etoolbox,refcount}
\usepackage{multicol}

\newcounter{countitems}
\newcounter{nextitemizecount}
\newcommand{\setupcountitems}{%
	\stepcounter{nextitemizecount}%
	\setcounter{countitems}{0}%
	\preto\item{\stepcounter{countitems}}%
}
\makeatletter
\newcommand{\computecountitems}{%
	\edef\@currentlabel{\number\c@countitems}%
	\label{countitems@\number\numexpr\value{nextitemizecount}-1\relax}%
}
\newcommand{\nextitemizecount}{%
	\getrefnumber{countitems@\number\c@nextitemizecount}%
}
\newcommand{\previtemizecount}{%
	\getrefnumber{countitems@\number\numexpr\value{nextitemizecount}-1\relax}%
}
\makeatother    
\newenvironment{AutoMultiColItemize}{%
	\ifnumcomp{\nextitemizecount}{>}{3}{\begin{multicols}{2}}{}%
		\setupcountitems\begin{itemize}}%
		{\end{itemize}%
		\unskip\computecountitems\ifnumcomp{\previtemizecount}{>}{3}{\end{multicols}}{}}


\bibliographystyle{ieeetr}  

\definecolor{codegreen}{rgb}{0,0.6,0}
\definecolor{codegray}{rgb}{0.5,0.5,0.5}
\definecolor{codepurple}{rgb}{0.58,0,0.82}
\definecolor{backcolour}{rgb}{0.95,0.95,0.92}

\lstdefinestyle{mystyle}{
	backgroundcolor=\color{backcolour},   
	commentstyle=\color{codegreen},
	keywordstyle=\color{magenta},
	numberstyle=\tiny\color{codegray},
	stringstyle=\color{codepurple},
	basicstyle=\footnotesize,
	breakatwhitespace=false,         
	breaklines=true,                 
	captionpos=b,                    
	keepspaces=true,                 
	numbers=left,                    
	numbersep=5pt,                  
	showspaces=false,                
	showstringspaces=false,
	showtabs=false,                  
	tabsize=2
}

\lstset{style=mystyle}

\begin{document}
	
\doublespacing

\begin{titlepage}

\newcommand{\HRule}{\rule{\linewidth}{0.5mm}} % Defines a new command for the horizontal lines, change thickness here

\center % Center everything on the page
 
%----------------------------------------------------------------------------------------
%   HEADING SECTIONS
%----------------------------------------------------------------------------------------

\textsc{\LARGE University of Melbourne}\\[1.5cm] % Name of your university/college
\textsc{\Large SWEN90004}\\[0.5cm] % Major heading such as course name
\textsc{\large Modelling Complex Software Systems}\\[0.5cm] % Minor heading such as course title

%----------------------------------------------------------------------------------------
%   TITLE SECTION
%----------------------------------------------------------------------------------------

\HRule \\[0.4cm]
{ \huge \bfseries Research Project}\\[0.4cm] % Title of your document
\HRule \\[1.5cm]

%----------------------------------------------------------------------------------------
%   AUTHOR SECTION
%----------------------------------------------------------------------------------------

\begin{minipage}{0.4\textwidth}
	\bfseries Group number : 47\\
	\bfseries Group member : Yang Zhang\\
	\bfseries Student ID : 956835
\end{minipage}\\[2cm]

% If you don't want a supervisor, uncomment the two lines below and remove the section above
%\Large \emph{Author:}\\
%John \textsc{Smith}\\[3cm] % Your name

%----------------------------------------------------------------------------------------
%   DATE SECTION
%----------------------------------------------------------------------------------------

{\large \today}\\[2cm] % Date, change the \today to a set date if you want to be precise

%----------------------------------------------------------------------------------------
%   LOGO SECTION
%----------------------------------------------------------------------------------------

\includegraphics[scale = 0.27]{logo.png}\\[1cm] % Include a department/university logo - this will require the graphicx package
 
%----------------------------------------------------------------------------------------

\vfill % Fill the rest of the page with whitespace

\end{titlepage}


\section{Background}
Wealth is always a  trade. Wealth distribution becomes an extremely hot topic for a long time, which contains two directions. The first direction focuses on how to increase the wealth gap between the poor and the rich. The second direction mainly and how to balance the wealth distribution. Wealth distribution is a giant and complex system in real world as there are uncountable factors to contribute variations. In our project, we aim to simulate a simplified wealth distribution system comparing to real world.\\

There are multiple reasons and motivations for us to analyze this topic. Firstly, it relates to our daily life. The more we know about the wealth, the better we could manipulate it such as daily usage or investment. In our model, people are divided into three classes, the poor, the middle and the rich. If we can summarized something helpful. It may contribute to others for further decision in the future, which is our motivation as well.\\

In our project, we aim to accomplish two part. The first part concentrates on replicating the original model provided by NetLogo. The basic attributes that contribute to wealth distribution listed blow.
  \begin{AutoMultiColItemize}
	\item num of people
	\item max vision
	\item max of metabolism
	\item min of life expectancy
	\item max of life expectancy
	\item percentage of best land
\end{AutoMultiColItemize}
This model is much simpler than real world model but the dialectical law of the development is the same, which states that gradually a small part of people possess majority of wealth while the whole system tends to be stable. However, it's still a complex system as a set of things work together as parts of a mechanism or an interconnecting network. The whole system consists of grain and people. People collect grains and compete with others.\\


The second part focuses on extensions. We add multiple variables and try to impact the wealth distribution.
   \begin{AutoMultiColItemize}
 	\item Heritage inheritance
 	\item Birth rate
 	\item Season
 	\item Reclamation
 \end{AutoMultiColItemize}
Further more, we are interested in finding a solution to reduce the wealth distribution.
\section{Design}
\subsection{Original Model}
The first task to rebuild original model is the whole world initialization, which contains grain and people initialization based on specific configuration. We assume that the whole world is a 2-dimension matrix. There are several properties for every single person, which are the same as model in NetLogo.
   \begin{itemize}
   	\item ID: identification of person.
   	\item Age: the age of person.
	\item Vision: determines how far a person could see.
	\item Metabolism: how much a person cost in every global clock.
	\item Life expectancy: how long a person stay alive
	\item Position: current position of person.
\end{itemize}
There are two behaviors for a person in every clock. Firstly, everyone tries to find a best land within vision. Then they cost some money which related to their predefined metabolism value with age increased by 1. When someone's age is larger than expected life expectancy, he will die and reborn with random values of properties.
\subsection{Extensions}
Based on original model, we are going to improve its complexity by adding multiple extensions. Here we list the extensions we added in this project. 
\begin{enumerate}
	\item Heritage inheritance\\
	In original model, people will discard all the money belonging to them when they are dead and generate money randomly when they reborn. In our extension, we introduce inheritance function for collecting their money when they reborn. Based on our assumption, it should be a way to speed up model development. 
	\item Birth rate\\
	
	\item Season\\
	The richness of land is fixed when the world is initialized in original model. We think that the circumstance could be more interesting if this factor varies. Therefore, the season function is come up. In our extended model, the richness of land varies from clock. We don't have any certain assumption of result with this season function.
	\item Reclamation\\
	Fixed size of land is another limitation in original model. We add reclamation function to extend the size of land as time goes. We believe it should be a good way to mitigate wealth distribution gap as the total available lands increase and there should be less competition comparing to before. 
\end{enumerate}
\section{Results}
\section{Discussion}



\end{document}